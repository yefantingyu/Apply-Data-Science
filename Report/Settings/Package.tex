\documentclass{article}
\documentclass[10pt,a4paper]{report}
\usepackage{tikz}
\usepackage{verbatim}
% not commen package
\usepackage[compact]{titlesec}
\usepackage[hang,flushmargin,stable]{footmisc}
\usepackage[nohyperlinks,nolist]{acronym}
\usepackage[T1]{fontenc}
\usepackage[usenames,dvipsnames,table]{xcolor}
\usepackage{amsfonts,amsmath,amssymb}
\usepackage{array}
\usepackage{booktabs}
\usepackage{csquotes}
\usepackage{enumitem}
\usepackage{eurosym}
\usepackage{fancyhdr}
\usepackage{lastpage}
\usepackage{lineno}
\usepackage{lmodern}
\usepackage{lscape}
\usepackage{makecell}
\usepackage{marginnote}
\usepackage{multirow}
\usepackage{pgfgantt}
\usepackage{pgfplots}
\usepackage{pgfplotstable}
\usepgfplotslibrary{polar}
\pgfplotsset{compat=1.12}
\usepackage{pifont}
\usepackage{sectsty}
\usepackage{sidecap}
\usepackage{soul} % for smarter (word-wrapping) underlining
\setul{1pt}{.4pt} % 1pt below contents
\usepackage{wrapfig}
\usepackage{xspace}
\usepackage{forest}
\usepackage{tikz}
\usetikzlibrary{arrows,shapes,chains}
\usepackage{hyperref}
\hypersetup{hypertex=true,
            colorlinks=true,
            linkcolor=blue,
            anchorcolor=blue,
            citecolor=blue}

% commen package
\usepackage{fancyhdr}
\usepackage{setspace}
\usepackage[utf8]{inputenc}
\usepackage{multirow}
\usepackage{subcaption}
\usepackage{enumitem}
\usepackage{listings}
\usepackage{float}
\usepackage{graphicx}
\usepackage{geometry}
\geometry{left=2.0cm,right=2.0cm,top=3.0cm,bottom=2.0cm}
\usepackage{times}
\usepackage{mathptmx}
\usepackage{color}
\usepackage[dvipsnames]{xcolor}
% HotFix from http://tex.stackexchange.com/a/300259/84485
% Version1 of titlesec is not compatible with the latest texlive. 
% Either the titlesec package must be updated, or the following HotFix used:
\usepackage{etoolbox}
\makeatletter
\patchcmd{\ttlh@hang}{\parindent\z@}{\parindent\z@\leavevmode}{}{}
\patchcmd{\ttlh@hang}{\noindent}{}{}{}
\makeatother

% Fonts
\usepackage{libertine}
\usepackage[scaled=.875]{gillius2}
\usepackage[libertine]{newtxmath}

% Size
\headheight=20pt
\setlength{\parindent}{0pt}
\titleformat*{\section}{\bold\Huge}
\titleformat*{\subsection}{\huge}
\titleformat*{\subsubsection}{\large}
\renewcommand{\headrulewidth}{0pt}
\renewcommand{\baselinestretch}{1.0} 

% Color
\definecolor{royalblue}{rgb}{0.0, 0.14, 0.4}
\allsectionsfont{\sffamily\bfseries\raggedright\color{royalblue}}
\let\oldfootnotesize\footnotesize
\newcommand{\highlight}[1]{\colorbox{gray!15}{\small{#1}}}
\newcommand{\marginLeft}[1]{\reversemarginpar\marginnote{\rotatebox{90}{\sffamily\bfseries\color{royalblue}\large{\phantom{j}#1}}}}
\newcommand{\marginRight}[1]{\normalmarginpar\hspace{0pt}\marginpar{\rotatebox{0}{\sffamily\color{gray}\normalsize{#1}}}}
\newcommand\colorrule{\vspace{4px}{\color{lightgray}\hrule}\vspace{4px}}

% Bibliography
%\bibliography{bibliography}
%\usepackage[style=chem-acs,doi=false]{biblatex}

% Headers and Footers
\renewcommand{\footnotesize}{\fontsize{10bp}{1em}\selectfont}
\renewcommand{\cite}{\autocite} % citations in footnotes
% Explicitly set footnote font size to match call (i.e., 8pt).
% Taken from http://tex.stackexchange.com/a/249422/84485
\makeatletter
\renewcommand\footnotesize{%
   \@setfontsize\footnotesize\@ixpt{8}%
   \abovedisplayskip 8\p@ \@plus2\p@ \@minus4\p@
   \abovedisplayshortskip \z@ \@plus\p@
   \belowdisplayshortskip 4\p@ \@plus2\p@ \@minus2\p@
   \def\@listi{\leftmargin\leftmargini
               \topsep 4\p@ \@plus2\p@ \@minus2\p@
               \parsep 2\p@ \@plus\p@ \@minus\p@
               \itemsep \parsep}%
   \belowdisplayskip \abovedisplayskip
}
\makeatother


\usepackage{color}
\definecolor{dkgreen}{rgb}{0,0.6,0}
\definecolor{gray}{rgb}{0.5,0.5,0.5}
\definecolor{mauve}{rgb}{0.58,0,0.82}


\lstdefinestyle{myPython}{% myPython是格式的名字
	frame = lrtb, % 显示边框
	captionpos=t, % 代码呈现的位置
	breaklines=true,%自动换行
	columns=fixed,  %
	%如果不加这一句,字间距就不固定,很丑,必须加
    basewidth=0.5em,
    showstringspaces=false,
    showspaces=false,
    flexiblecolumns,
	language=Python,
	aboveskip=3mm,
	belowskip=3mm,
	showstringspaces=false,
	columns=flexible,
	numberstyle=\small\color{red},
	basicstyle={\Monaco},
	keywordstyle={\color{blue}\Monaco},
	commentstyle={\color{dkgreen}\Monaco},
	stringstyle={\color{mauve}\Monaco},
	breaklines=true,
	breakatwhitespace=true,
	tabsize=3
}


\definecolor{codegreen}{rgb}{0,0.6,0}
\definecolor{codegray}{rgb}{0.5,0.5,0.5}
\definecolor{codepurple}{HTML}{C42043}
\definecolor{backcolour}{HTML}{F2F2F2}
\lstdefinestyle{MySQL}{
    language        =   SQL, % 语言选Python
    basicstyle      =   \Monaco,
    % numberstyle     =   \ttfamily,
    keywordstyle    =   \color{blue}\Monaco,
    keywordstyle    =   [2] \color{teal},
    stringstyle     =   \color{magenta},
    commentstyle    =   \color{red}\ttfamily,
    breaklines      =   true,   % 自动换行,建议不要写太长的行
    columns         =   fixed,  % 如果不加这一句,字间距就不固定,很丑,必须加
    basewidth       =   0.5em,
}
\lstset{
    basicstyle          =   \Monaco,          % 基本代码风格
    keywordstyle        =   \fontspec{Monaco},          % 关键字风格
    commentstyle        =   \rmfamily\itshape,  % 注释的风格,斜体
    stringstyle         =   \ttfamily,  % 字符串风格
    flexiblecolumns,                % 别问为什么,加上这个
    % numbers             =   left,   % 行号的位置在左边
    showspaces          =   false,  % 是否显示空格,显示了有点乱,所以不现实了
    numberstyle         =   \ttfamily,    % 行号的样式,小五号,tt等宽字体
    showstringspaces    =   false,
    captionpos          =   t,      % 这段代码的名字所呈现的位置,t指的是top上面
    frame               =   lrtb,   % 显示边框
    style = MySQL
}




\def\BibTeX{{\rm B\kern-.05em{\sc i\kern-.025em b}\kern-.08em
    T\kern-.1667em\lower.7ex\hbox{E}\kern-.125emX}}
\usepackage{cite}


\pagestyle{fancy}                   % 设置页眉页脚
\lhead{DATA70202}   %页眉左侧显示页数                 
% \chead{}                                  %页眉中
\rhead{Group 10}                         %章节信息                       
\cfoot{\thepage}      
\usepackage[utf8]{inputenc}
